\documentclass[article,shortnames]{jss}

\usepackage{amsmath}
\usepackage[utf8x]{inputenc}

\newcommand{\TODO}[1]{{\color{red} TODO: #1}}

\graphicspath{ {./img/} }
\setkeys{Gin}{width=2.75in,keepaspectratio}


%%%%%%%%%%%%%%%%%%%%%%%%%%%%%%
%% declarations for jss.cls %%%%%%%%%%%%%%%%%%%%%%%%%%%%%%%%%%%%%%%%%%
%%%%%%%%%%%%%%%%%%%%%%%%%%%%%%

\author{Vincent Dorie\\New York University}
\title{\pkg{blme}: Profiled Maximum A Posteriori Estimation of
  Bayesian Linear Mixed Effects Models in \proglang{R}}

\Plainauthor{Vincent Dorie}
\Plaintitle{blme: Profiled Maximum A Posteriori Estimation of
  Bayesian Linear Mixed Effects Models in R}
\Shorttitle{\pkg{blme}: Bayesian Linear Mixed Effects in \proglang{R}}

\Abstract{
  The popular \proglang{R} package \pkg{lme4} allows the fast fitting
  of linear and generalized linear mixed effect models through the use
  of a profiled likelihood function. This article details
  Bayesian, or penalized-likelihood, extensions that
  enable a wider class of models to be fit while preserving
  computational efficiency. The package \pkg{blme}, also for
  \proglang{R}, implements a wide variety of these models while
  inheriting functionality from \pkg{lme4}. Projected uses include the
  incorporation of substantive prior information, regularization of
  fixed effects, meta analysis, and the use of weakly informative
  priors to obtain non-degenerate random effect covariance matrix estimates.
}
\Keywords{Bayesian, penalized likelihood, mixed effects, \pkg{blme}, \pkg{lme4}, \proglang{R}}
\Plainkeywords{Bayesian, penalized likelihood, mixed effects, blme, lme4, R}

%% publication information
%% NOTE: Typically, this can be left commented and will be filled out by the technical editor
%% \Volume{50}
%% \Issue{9}
%% \Month{June}
%% \Year{2012}
%% \Submitdate{2012-06-04}
%% \Acceptdate{2012-06-04}

%% The address of (at least) one author should be given
%% in the following format:
\Address{
  Vincent Dorie\\
  Department of Humanities and Social Sciences in the Professions \\
  Postdoctoral fellow at the Center for the Promotion of Research Involving
  Innovative Statistical Methodology \\
  New York University \\
  246 Greene St, Rm 318E \\
  New York, NY 10003, USA
  E-mail: \email{vjd4@nyu.edu}\\
  URL: \url{http://buildmeawebpagesomeday/}
}

%% \usepackage{Sweave.sty}

%% end of declarations %%%%%%%%%%%%%%%%%%%%%%%%%%%%%%%%%%%%%%%%%%%%%%%


\begin{document}


\section[Introduction]{Introduction}

History, references for lmm, glmms. AKA.

Proof of popularity? Popularity of lme4?

lmms lme4 profiles; analysis of profiling technique shows what
additional models can fit

map, penalized likelihood; meta analysis, ridge regression; boundary estimates

competing packages

things you can do, not whether or not you should do them

\section[Profiled likelihood and lme4]{Profiled likelihoods and linear
  mixed models}

In the next few sections, we derive Bayesian extensions / likelihood penalties that
can be applied to the linear mixed model fit by \pkg{lme4} with little to no
additional computation cost. To do so, we first trace through the
calculations that \pkg{lme4} uses to profile the likelihood and avoid
costly numeric optimization techniques, after which the modifications
are introduced. A few relevant concepts, such as common scale parameters and
REML estimation are introduced as well. When comparing two equations where one has
been altered to incorporate a prior, the key differences are highlighted in
{\color{blue}blue} text.

We also note that this section applies only
to {\em linear} mixed models. It is the approach of profiling that
makes adding priors difficult, and as generalized linear mixed models
in \pkg{lme4} are fit without profiling, priors are considerably easier to
implement. They are briefly discussed in section \ref{sec:glmms}.

\subsection[lme4 linear mixed model specification]{\pkg{lme4} linear mixed model specification}

In concise, matrix notation the linear mixed effect model fit by \pkg{lme4} can be written as

\begin{align}
  \label{eq:conditionalModel}
  y\mid b & \sim \mathcal{N}(X\beta + Zb, \sigma^2 I_n), \nonumber \\
  b & \sim \mathcal{N}(0, \sigma^2\Sigma),
\end{align}

\noindent where $X$ and $Z$ are matrices of suitable dimension,
$\beta$ are the fixed effects, $b$ are the random effects, $\sigma^2$
is the residual variance or ``common scale'' parameter, and $\Sigma$
is the covariance of the random
effects. Furthermore, $Z$ and $\Sigma$ are sparse, as $Z$ serves to
select the specific random effects that are applicable to an
observation and the random effects are independent between groups and
grouping factors. The structure of $\Sigma$ is irrelevant to maximum
likelihood so we ignore it for the moment, although it is explored
in-depth in section \ref{sec:bmath:covPriors} when random effect
covariance priors are considered. As $b$ is latent, integrating it out yields the marginal
model wherein

\begin{equation}
  \label{eq:marginalModel}
  y \sim \mathcal{N}(X\beta, \sigma^2 I_n + \sigma^2 Z \Sigma Z^\top),
\end{equation}

\noindent subject to the constraint that $\Sigma$ is positive
semi-definite. The parameters are $\beta$, $\sigma^2$, and $\Sigma$.

A reader familiar with generalized least-squares problems might notice
that, for a given value of $\Sigma$, the maximum likelihood estimates
of $\beta$ and $\sigma^2$ are easy to obtain. That is indeed the case,
but rather than directly optimize the associated likelihood and
repeatedly invert a matrix of dimension equal to the sample size, the
authors of \pkg{lme4} employ a series of profiling steps
that exploit the sparsity of the problem. We retrace their solution
here, as it has implications for what priors can be imposed without
adding substantial computation.

\subsection{Common scale}
\label{sec:math:commonScale}

Here we take a brief aside to discuss the implications of modeling the
random effects on the same scale as the observations, that is the
appearance of $\sigma^2$ on the second line of equation
\ref{eq:conditionalModel}. As will soon be shown, the choice greatly
simplifies optimization. However, for the purposes of the extending the
model it essentially requires that priors also be placed on this
common scale if the efficiency gains of \pkg{lme4} are to be preserved.

In the absence of substantive prior knowledge, there do not seem to be
any strong reasons to want to influence the model in ways that aren't
modulo the residual variance. However, when such knowledge does
exist incorporating it sometimes comes at the cost of mathematical
convenience and profiling must be abandoned. Exceptions will be noted as they occur.

\subsection{Restricted/residual maximum likelihood}

As a final preliminary, we consider restricted maximum likelihood estimation, abbreviated
REML. REML is commonly used as to unbias the estimate of the conditional
variance. In one sense, it takes into account the cost of estimating
the fixed effects, although for our purposes it is sufficient to know
that it also arises from integrating out the fixed effects from the
likelihood using a flat prior. Since this assumption is rarely made
explicit, we refer to REML as an estimation procedure and not a
model. Consequently, in REML estimation an objective
function/criterion is maximized instead of a likelihood. This function
we label `$q$', and refer to proper probability densities by
`$p$'.

\subsection{Definition of profiled likelihood}

\begin{figure}[t]
  \centering \includegraphics[height=2.75in, keepaspectratio = TRUE]{profileContour}
  \caption[Profiled likelihood illustration]{Illustration of a profiled likelihood. The left graph shows
  the likelihood for a simple linear model mixed model with a varying intercept as a function of $\sigma$ and
  $\sqrt\Sigma$, which we abbreviate as $\sigma_b$ ($\beta$ has already been maximized). The gray line corresponds
  to the maximum in $\sigma$ as a function of $\sigma_b$. Following the contour along this line produces the profiled
  likelihood of $\sigma_b$ on the right. The profiled
  likelihood goes through the joint mode, so that maximizing it is
  sufficient to maximize the likelihood.}
  \label{fig:profileContour}
\end{figure}

A profiled likelihood is a function over several parameters, some of which
have been optimized analytically. Once a maximizer for one or more
parameters has been derived, these estimators can be plugged back into the
likelihood and the resulting equation optimized instead.

Figure \ref{fig:profileContour} demonstrates this approach for a simple
linear mixed effects model with only a varying intercept. The likelihood is a function of the three
parameters $\beta$, $\sigma^2$, and $\Sigma$ - all of which in this case are
scalars. The estimator of $\beta$ that maximizes the likelihood is a
function of the two variance parameters and has an explicit solution. Plugging
this into the likelihood yields a first-stage profiled
function. Similarly, the maximizer of this function with respect to
$\sigma^2$ can be obtained as a function of just $\Sigma$. These two steps
produce a profiled likelihood that is a function only of the variance of
the random effects, here reducing the problem from three parameters to just one.

\subsection{Profiled likelihood derivation}

One of the key features of \pkg{lme4} for linear mixed models is that
it performs profiled optimization by analytically maximizing the
likelihood in first the fixed
effects and subsequently the residual variance. Only the covariance of the random
effect requires numeric techniques. In this section, we recreate
the derivation of the profiled likelihood so that it is possible to
demonstrate which Bayesian extensions fit in this two-step profiling scheme.

We start with the joint density of the response and random
effects. The random effects must be integrated from this equation to
obtain the likelihood, but
we first manipulate it so
as to simplify subsequent optimization. To be precise, denote $N$ as the length of $y$, or the
total number of observations. Let $Q$ be the total number of random effects
and $P$ the number of fixed effects. Then,

\begin{equation*}
  p(y, b; \beta, \sigma^2, \Sigma) = (2\pi\sigma^2)^{-(N+Q)/2}
  |\Sigma|^{-1/2} 
  \exp\left\{-\frac{1}{2\sigma^2}\left[ \|y - X\beta - Zb\|^2 + b^\top
      \Sigma^{-1} b\right]\right\}.
\end{equation*}

As we intend to integrate out $b$, we can make the change of variables
to spherical random effects. If $\Lambda\Lambda^\top = \Sigma$ is
a left-Cholesky factorization of $\Sigma$, then for $b = \Lambda u, u
\sim \mathcal{N}(0, I_Q)$ we have

\begin{align}
  \label{eq:jointDensity}
  p(y, u; \beta, \sigma^2, \Lambda) & = (2\pi\sigma^2)^{-(N+Q)/2} 
  \exp\left\{-\frac{1}{2\sigma^2}\left[\|y - X\beta - Z\Lambda u\|^2 +
      \|u\|^2\right]\right\}, \nonumber \\
    & = (2\pi\sigma^2)^{-(N+Q)/2} 
    \exp\left\{-\frac{1}{2\sigma^2}\left\|\begin{bmatrix}y \\ 0 \end{bmatrix}
    -\begin{bmatrix}Z\Lambda & X \\ I_Q &
      0 \end{bmatrix}\begin{bmatrix} u \\ \beta \end{bmatrix}\right\|^2\right\}.
\end{align}

At this point it is possible to identify the maximum likelihood
estimate of $\beta$. For the moment conceptualizing
$\beta$ as a random variable, the preceeding equation has a quadratic form in the
exponential term for the vector $(u, \beta)$. In turn, this
defines the two as
jointly-normal and implies that the marginal mode for $\beta$ is the
same as the joint mode. As we now detail, the MLE is obtained by finding the mode of
this joint distribution in $u$ and $\beta$, ``conditioning'' on
$\beta$, and integrating out $u \mid \beta$.

Denote $\tilde{u}$ and $\tilde{\beta}$ as the modes of this joint
distribution. They are obtained in a nuanced fashion that exploits
sparse matrix decompositions and is reproduced for completeness in
appendix \ref{sec:appendix:jointMode}. The decomposition that is used
is given by the blocks of

\begin{align*}
  \begin{bmatrix} L_Z & 0 \\ L_{ZX} & L_X \end{bmatrix}
  \begin{bmatrix} L_Z^\top & L_{ZX}^\top \\ 0 & L_X^\top \end{bmatrix}
  & = 
  \begin{bmatrix}\Lambda^\top Z^\top & I_Q \\ X^\top & 0 \end{bmatrix}
  \begin{bmatrix}Z \Lambda & X \\ I_Q & 0 \end{bmatrix},  \\
  & =  \begin{bmatrix} \Lambda^\top Z^\top Z \Lambda + I_Q & \Lambda^\top Z^\top X \\
  X^\top Z \Lambda & X^\top X \end{bmatrix}.
\end{align*}

\noindent Specifically,

\begin{align}
  \label{eq:blockDecomp}
  L_Z L_Z^\top & = \Lambda^\top Z^\top Z \Lambda + I_Q, \nonumber \\
  L_{ZX} & = X^\top Z \Lambda L_Z^{-\top}, \nonumber \\
  L_X L_X^\top & = X^\top X - L_{ZX} L_{ZX}^\top.
\end{align}

As $Z$ is sparse and $\Lambda$ is block-diagonal, $L_Z$ can be
computed and stored efficiently. While we typically omit the
dependence, it is important to note that these matrices depend on
$\Lambda$. Utilizing this decomposition and the
joint modes, we can rewrite the joint distribution by completing the square:

\begin{multline*}
  p(y, u; \beta, \sigma^2, \Lambda) = (2\pi \sigma^2)^{-(N+Q)/2}
  \exp\left\{-\frac{1}{2\sigma^2} \left( \begin{bmatrix} u - \tilde{u} \\
        \beta - \tilde{\beta}\end{bmatrix}^\top
      \begin{bmatrix} L_Z & 0 \\ L_{ZX} & L_X \end{bmatrix}
      \begin{bmatrix} L_Z^\top & L_{ZX}^\top \\ 0 & L_X^\top \end{bmatrix}
      \begin{bmatrix} u - \tilde{u} \\ \beta - \tilde{\beta} \end{bmatrix} + \right.\right. \\
%%
      \left.\left.\left\|\begin{bmatrix} y \\ 0 \end{bmatrix} - \begin{bmatrix} Z
          \Lambda & X \\ I_Q & 0 \end{bmatrix} \begin{bmatrix} \tilde{u} \\
          \tilde{\beta}\end{bmatrix}\right\|^2\right) \right\}.
\end{multline*}

Conditioning on $\beta$ simply rotates its dependence with $u$ into
its mean, so that writing $\mu_{u\mid\beta} = \tilde{u} -
L_Z^{-\top}L_{ZX}^\top(\beta - \tilde{\beta})$ we have:

\begin{multline*}
  p(y, u; \beta, \sigma^2, \Lambda) = (2\pi \sigma^2)^{-(N+Q)/2}
  \exp\left\{-\frac{1}{2\sigma^2} \left( \begin{bmatrix} u -
        \mu_{u\mid \beta} \\
        \beta - \tilde{\beta} \end{bmatrix}^\top
      \begin{bmatrix} L_Z & 0 \\ 0 & L_X \end{bmatrix}
      \begin{bmatrix} L_Z^\top & 0 \\ 0 & L_X^\top \end{bmatrix}
      \begin{bmatrix} u - \mu_{u \mid \beta} \\ \beta - \tilde{\beta} \end{bmatrix} + \right.\right. \\
%%
      \left.\left.\left\|\begin{bmatrix} y \\ 0 \end{bmatrix} - \begin{bmatrix} Z
          \Lambda & X \\ I_Q & 0 \end{bmatrix} \begin{bmatrix} \tilde{u} \\
          \tilde{\beta}\end{bmatrix}\right\|^2\right) \right\}.
\end{multline*}

At this point, it is trivial to integrate out $u$ as it has the
density of Gaussian random variable which we can factor in the form
of $p(u\mid \beta)p(\beta)$. Performing this integration, we obtain the rearranged marginal likelihood

\begin{multline}
  p(y; \beta, \sigma^2, \Lambda) = (2\pi\sigma^2)^{-N/2} |L_Z|^{-1}
  \exp\Bigg\{-\frac{1}{2\sigma^2}\Bigg[(\beta - \tilde{\beta})^\top
  L_XL_X^\top (\beta - \tilde{\beta}) + \\ 
  \left.\left.\left\| \begin{bmatrix} y \\
          0 \end{bmatrix} - \begin{bmatrix} Z \Lambda & X \\ I_Q & 0 \end{bmatrix}
        \begin{bmatrix} \tilde{u} \\ \tilde{\beta} \end{bmatrix} \right\|^2\right]\right\}.
\label{eq:likelihood}
\end{multline}

Having the likelihood in this form makes the profiling steps straightforward. The maximum
likelihood estimator of $\beta$ is $\hat\beta = \tilde{\beta}$, which
is also the joint mode. Plugging this back in gives us the first stage
profiled equation

\begin{equation*}
  p(y; \hat\beta, \sigma^2, \Lambda) = (2\pi\sigma^2)^{-N/2}|L_Z|^{-1} 
  \exp\left\{-\frac{1}{2\sigma^2} \left\| \begin{bmatrix} y \\
        0 \end{bmatrix} - \begin{bmatrix} Z \Lambda & X \\ I_Q & 0 \end{bmatrix}
      \begin{bmatrix} \tilde{u} \\ \tilde{\beta} \end{bmatrix} \right\|^2\right\}.
\end{equation*}
  
From this, maximizing in $\sigma^2$ yields the MLE $\hat{\sigma}^2 = \frac{1}{N}\left[ \|y - X\hat{\beta}
  - Z\Lambda \tilde{u} \|^2 + \|\tilde{u}\|^2\right]$. Utilizing this in turn produces the fully profiled likelihood:

\begin{equation}
  p(y;\hat{\beta}, \hat{\sigma}^2, \Lambda) = \left(2\pi
  \hat{\sigma}^2(\Lambda)\right)^{-N/2} |L_Z(\Lambda)|^{-1}e^{-N/2}.
\label{eq:profiledLikelihood}
\end{equation}

REML estimation proceeds by integrating $\beta$ from the likelihood as a Gaussian
random variable rather than by maximization. The REML objective function is

\begin{equation}
  q(y; \sigma^2, \Lambda) = (2\pi\sigma^2)^{-(N-P)/2}
  |L_Z|^{-1}|L_X|^{-1} \exp\left\{-\frac{1}{2\sigma^2}
    \left\|\begin{bmatrix} y \\ 0 \end{bmatrix} - \begin{bmatrix}
        Z\Lambda & X \\ I_Q & 0 \end{bmatrix} \begin{bmatrix} \tilde{u} \\ 
        \tilde{\beta} \end{bmatrix} \right\|^2\right\}.
\end{equation}

\noindent The REML estimate of $\sigma^2$ is $\hat\sigma^2_{\mathrm{RE}} = \frac{1}{N-P}\left[ \|y
  - X \tilde{\beta} - Z \Lambda \tilde{u} \|^2 +
  \|\tilde{u}\|^2\right]$, while the profiled REML objective function is

\begin{equation}
  q(y; \hat{\sigma}^2, \Lambda) = \left(2\pi
    \hat{\sigma}^2_{\rm RE}(\Lambda)\right)^{-(N-P)/2}|L_Z(\Lambda)|^{-1}|L_X(\Lambda)|^{-1}e^{-(N-P)/2}.
\label{eq:remlObjective}
\end{equation}

\section{Profiled posteriors for linear mixed models}

In this section we describe Bayesian extensions to linear mixed models
which can be profiled in fashion similar to the two-step approached
used for the likelihood. Arbitrary
priors can be applied and the posterior mode found instead, but doing
so may dramatically increase the
number of parameters that require numeric optimization.

\subsection{Preliminaries}
\label{sec:bmath:prelim}

As priors are successively applied to model components it becomes
important to clearly define the quantity that is being estimated. For example, when placing
a prior over the fixed effects ($\beta$) but not the other parameters, the
traditional Bayesian estimand is the posterior distribution $p(u, \beta | y; \sigma^2, \Sigma)$
while the variance components ($\sigma^2$ and $\Sigma$) are hyperparameters of this
distribution. Point estimation in this setting
corresponds to estimating the posterior means of $u$ and $\beta$, that is
$\E\left[ (u, \beta) \mid y; \hat{\sigma}^2, \hat\Sigma\right]$.

Conversely, the ``likelihood'' can be redefined. Once $\beta$ has
been modeled, classically it should be integrated out from
the joint distribution to yield the marginal distribution, $p(y; \sigma^2,
\Sigma)$. In a sense, the fixed effects become random effects with a
known distribution. For the fixed effects there is some tradition in
favor of taking this integral - as mentioned the REML
estimates can be derived by applying and averaging $\beta$ over a flat
prior. However, little similar precedent exists for the covariance of
the random effects, so while it may be ideologically pure to do so, it
also may confound expectations.

To be consistent, we instead adopt the perspective of the penalized
likelihood, or regularization. For example, putting a prior on $\beta$
is equivalent to maximizing the posterior density
$p(\beta \mid y; \sigma^2, \Sigma)$ to find $p(\hat{\beta} \mid y;
\hat{\sigma}^2, \hat{\Sigma})$. While this hybrid quantity may be unusual to the
Bayesian, it represents a stop-gap on the way to a full posterior mode
obtained by placing priors over all model components. Because
\pkg{lme4} offers REML estimation, when it makes sense to do so we
investigate that integral as well.

A related issue concerning parameterization arises when
moving from prior to posterior. Continuing in the perspective of the
penalized likelihood, we assume that if a prior is specified on, say, the inverse of the residual variance,
the posterior mode of the inverse of the residual variance is desired. In
other words, a prior once specified is applied directly as a penalty
function, and the Jacobian that results from a change of
variables to a canonical form is simply ignored. This can
produce some confusion because a parameter may be alternatively
reported as a variance and a standard deviation while the posterior
mode that is calculated may be of neither. Nevertheless, we find this
preferable to finding the posterior modes under implicit transformations and having the awkwardness
of a variance not being the square of its associated standard deviation.

\subsection{Fixed effect priors}
\label{sec:bmath:fixefPrior}

Gaussian priors on the fixed effects require little effort to include
as they can be treated as pseudo data. The program for optimization follows
almost identical steps as that for the first stage of the
likelihood. If we assume that $\beta \mid \sigma^2 \sim
\mathcal{N}(0, \sigma^2\Sigma_\beta)$ with $\Sigma_\beta$ known,
positive definite, and
having the decomposition $\Sigma_\beta = L_\beta L_\beta^\top$, then
the joint density from equation \ref{eq:jointDensity} becomes

\begin{align*}
  p(y, u, \beta; \sigma^2, \Sigma) & = (2\pi\sigma^2)^{-(N+Q+{\color{blue}P})/2}
  |\Sigma_\beta|^{-1/2} \exp\left\{-\frac{1}{2\sigma^2}\left[\|y - X\beta - Z \Lambda u
      \|^2 + \|u\|^2 + {\color{blue}\beta^\top\Sigma_\beta^{-1}\beta}\right]\right\}, \\
  & = (2\pi\sigma^2)^{-(N+Q+ {\color{blue}P})/2}|\Sigma_\beta|^{-1/2} \exp\left\{-\frac{1}{2\sigma^2}
    \left\|\begin{bmatrix}y \\ 0 \\ 0\end{bmatrix} -
      \begin{bmatrix} Z \Lambda & X \\ {\color{blue} 0} & {\color{blue}L_\beta^{-1}} \\ I_Q & 0 \end{bmatrix}
      \begin{bmatrix} u \\ \beta \end{bmatrix}\right\|^2\right\}.
\end{align*}

Note the occurrence of $\sigma^2$ when modeling $\beta$. In this
case, use of the common scale factor allows us to proceed almost
exactly as before:
$L_XL_X^\top$ changes from $X^\top X - L_{ZX}L_{ZX}$ to $X^\top X + \Sigma_\beta^{-1} -
L_{ZX}L_{ZX}^\top$ and 
the degrees of freedom for $\sigma^2$ increase from $N$ to $N +
P$. Once $L_X$ has been redefined, the modes of this function,
$\tilde{u}$ and $\tilde{\beta}$, are calculated with no other
changes. The marginal posterior used as an objective function for optimization
is given by

\begin{multline*}
  p(\beta \mid y; \sigma^2, \Lambda) \propto
  (\sigma^2)^{-(N+P)/2}|L_Z|^{-1} \times \\
  \exp\left\{-\frac{1}{2\sigma^2} \left[ (\beta - \tilde\beta)^\top
      L_XL_X^\top (\beta - \tilde\beta) + \left\|\begin{bmatrix}y \\ 0 \\ 0\end{bmatrix} -
      \begin{bmatrix} Z \Lambda & X \\ 0 & L_\beta^{-1} \\ I_Q & 0 \end{bmatrix}
      \begin{bmatrix} \tilde{u} \\ \tilde\beta \end{bmatrix}\right\|^2\right] \right\}.
\end{multline*}

\noindent The maximum a posteriori estimate of $\beta$ is
$\tilde\beta$, and the posterior at its mode has maximal values for
the hyperparameter value $\hat{\sigma}^2 =
\frac{1}{N+P}\left[\|y - X\hat{\beta} - Z\Lambda\tilde{u}\|^2 +
  \|\tilde{u}\|^2 + \|\tilde L_{\beta}^{-1}
  \tilde\beta\|^2\right]$. The profiled posterior at its mode is:

\begin{equation*}
  p(\hat{\beta} \mid y; \hat{\sigma}^2, \Lambda) \propto
  (\hat{\sigma}^2(\Lambda))^{-(N+P)/2} |L_Z(\Lambda)|^{-1}.
\end{equation*}

Now correctly denoted a likelihood, the REML procedure of integrating
out $\beta$ yields the function:

\begin{equation*}
  p(y;\sigma^2, \Lambda) = (2\pi\sigma^2)^{-N/2}|\Sigma_\beta|^{-1/2}
  |L_Z|^{-1} |L_X|^{-1}   \exp\left\{-\frac{1}{2\sigma^2} \left\|\begin{bmatrix}y \\ 0 \\ 0\end{bmatrix} -
      \begin{bmatrix} Z \Lambda & X \\ 0 & L_\beta^{-1} \\ I_Q & 0 \end{bmatrix}
      \begin{bmatrix} \tilde{u} \\ \tilde\beta \end{bmatrix}\right\|^2\right\}.
\end{equation*}

The estimator $\hat{\sigma}^2_{\rm RE}$ is the same as the previous case, except for being scaled by
$N$ instead of $N+P$. The profiled likelihood/REML objective function is:

\begin{equation*}
  p(y; \hat{\sigma}^2, \Lambda) =
  (2\pi\hat{\sigma}^2_{\rm RE}(\Lambda))^{-N/2}|\Sigma_\beta|^{-1/2} |L_Z(\Lambda)|^{-1}|L_X(\Lambda)|^{-1}
  e^{-N/2}.
\end{equation*}

Had we instead chosen to incorporate substantive prior knowledge,
i.e. assumed $\beta \sim \mathcal{N}(0, \Sigma_\beta)$ without utilizing
the common scale, the least-squares problem is solvable but then
depends on the value of $\sigma^2$. In turn, this needs to be added to
the set of parameters optimized numerically. Specifically, the joint
density (equation \ref{eq:jointDensity}) becomes:

\begin{equation*}
  p(y, u, \beta; \sigma^2, \Sigma) = (2\pi\sigma^2)^{-(N+Q+P)/2}
  |\Sigma_\beta|^{-1/2}
  \exp\left\{-\frac{1}{2\sigma^2}\left\|
      \begin{bmatrix} y \\ 0 \\ 0 \end{bmatrix} - 
      \begin{bmatrix} Z\Lambda & X \\ 0 & {\color{blue}\sigma}L_\beta^{-1} \\
        I_Q & 0 \end{bmatrix}
      \begin{bmatrix} u \\ \beta \end{bmatrix}
    \right\|^2\right\}.
\end{equation*}

$L_XL_X^\top$ would then be computed as factors of $X^\top X + {\color{blue}
  \sigma^2}\Sigma_\beta^{-1} - L_{ZX}L_{ZX}^\top$ so that the joint
modes $\tilde{u}(\sigma^2, \Lambda)$ and $\tilde{\beta}(\sigma^2, \Lambda)$ are properly
written as depending on the common scale as well as the covariance of
the random effects.

More complicated priors, such as $t$ distributions, require adding
$\beta$ to the set of parameters for numeric optimization.

\subsection{Common scale priors}
\label{sec:bmath:residPrior}

\begin{table}
  \begin{center}
  \begin{tabular}{lcccc}
    case & function & norm const & $\nu$ & $S^2$ \\ \hline
    $\beta$ unmodeled, ML & $p(y; \hat\beta, \sigma^2, \Lambda)$ &
    $(2\pi)^{-\nu/2}$ & $N$ & $\text{PRSS}$ \\
    $\beta$ unmodeled, REML & $q(y; \sigma^2, \Lambda)$ &
    $(2\pi)^{-\nu/2}|L_X|^{-1}$ & $N-P$ & $\text{PRSS}$ \\
    $\beta \sim \mathcal{N}(0, \sigma^2\Sigma_\beta)$, MAP & $p(\hat\beta \mid y; \sigma^2,
    \Lambda)$ & NA & $N+P$ &  $\text{PRSS} + \|L_\beta^{-1}\beta\|^2$ \\
    $\beta \sim \mathcal{N}(0, \sigma^2\Sigma_\beta)$, ML & $p(y;
    \sigma^2, \Lambda)$ & $(2\pi)^{-\nu/2}|L_X|^{-1}|\Sigma_\beta|^{-1/2}$ & $N$
    & $\text{PRSS} + \|L_\beta^{-1}\beta\|^2$
  \end{tabular}
  \end{center}
  \caption{The variable components of equation
    \ref{eq:commonScaleObjective} broken down by the choice of model
    for the fixed effects. `$\text{PRSS}$' stands for the base-model penalized residual
    sum of squares, $\|y - X\hat{\beta} -
    Z\Lambda\tilde{u}\|^2 + \|\tilde{u}\|^2$. The first and third rows
    correspond to straight maximization, while the second and fourth
    arise from the REML procedure of integrating out $\beta$.}
  \label{tab:commonScalePriorSpecifics}
\end{table}

Under the assumption that the fixed effects can be successfully profiled out
without depending on the common scale, we have a first-stage profiled
equation of the form

\begin{equation}
  f(y; \sigma^2, \Lambda) \propto (\sigma^2)^{-\nu/2}|L_Z|^{-1}
  \exp\left\{-\frac{1}{2\sigma^2}S^2\right\}.
\label{eq:commonScaleObjective}
\end{equation}

\noindent Table \ref {tab:commonScalePriorSpecifics} enumerates the
possible instantiations of this equation for the cases we have
considered. Taking a logarithm, derivative with respect to
$\sigma^2$, and rescaling yields a linear equation with a maximizer in
$\hat{\sigma}^2$ that is $\frac{1}{\nu}S^2$.

Unlike the case for the fixed effects, application of an independent
prior to $\sigma^2$ can proceed directly from this equation without
reiterating its derivation. A conjugate prior, that is an inverse
gamma on the scale of $\sigma^2$, will again yield a linear optimization
problem. Specifically, if we use a shape parameter of $a$ and a scale
of $b$, then $\hat\sigma^2 = \frac{1}{\nu/2 + a + 1}(S^2/2 + b)$.

More complicated results follow from other variants of the gamma
family. If we apply a gamma distribution to $\sigma^2$, not its
inverse, and again with shape $a$ and scale $b$, we transform the base
function objective function into the form

\begin{equation*}
  g(y; \sigma^2, \Lambda) \propto(\sigma^2)^{-\nu/2 + a -
    1}|L_Z|^{-1}\exp\left\{-\frac{1}{2\sigma^2}S^2 - \frac{\sigma^2}{b}\right\}.
\end{equation*}

A logarithm and a derivative with respect to $\sigma^2$ produces a
quadratic problem. Taking the non-negative square root yields

\begin{equation*}
  \hat{\sigma}^2 = \frac{b}{2}\left[ \sqrt{(\nu/2 - a + 1)^2 + 2 S^2 /
      b} - (\nu / 2 - a + 1)\right].
\end{equation*}

It is also possible to apply an inverse gamma to $\sigma$ as
a standard deviation and obtain a quadratic problem. The objective
function and its mode under a gamma prior with shape $a$ and scale $b$
are given by:

\begin{align*}
  g(y, \sigma^2, \Lambda) & = (\sigma^2)^{-(\nu + a + 1)/2}
  \exp\left\{-\frac{1}{2\sigma^2}S^2 - \frac{b}{\sigma}\right\}, \\
  \hat\sigma & = \frac{b + \sqrt{b^2 + 4(\nu + a + 1)S^2}}{2(\nu + a + 1)}.
\end{align*}

Finally, we also note that it is possible, and often desirable, to use
a point-mass prior for the residual variance. For example, this arises in
meta-analyses which can be written as weighted linear mixed models
having a residual variance of 1. While maintaining the same general
profiling procedure, little effort is
required to simply plug-in a specific value for $\sigma^2$ instead of
estimating it.

In summary, for the priors on $\beta$ that have maximizers independent
of $\sigma^2$, the application an a) inverse-gamma prior on
$\sigma^2$, b) gamma on $\sigma^2$, or c) inverse-gamma on $\sigma$
yield linear, quadratic, or quadratic optimization problems
respectively. Point priors require no optimization at all. When these conditions are satisfied
we proceed as before, first solving for the maximum in $(u, \beta)$ as before, then using these
to find the maximum in $\sigma^2$ if necessary, and finally by plugging these values into a
likelihood/objective function such as equation \ref{eq:likelihood}. The maximizer in $\sigma^2$ may no longer
be proportional to the exponential term so that the resulting equation may have a more
complex form than, say the profiled likelihood (equation \ref{eq:profiledLikelihood}),
however this does not come with increased computational complexity.

Finally we note that it is also possible to apply more complicated priors and maintain the
profiling scheme so long as a unique mode for $\sigma^2$ exists. An
inner-loop single parameter optimization or, equivalently, a root
finding algorithm can serve the role of explicit derivations without
greatly increasing running time. The final case considered in section
\ref{sec:bmath:fixefPrior}, with a prior for the fixed effects having
a covariance specified on an absolute scale, can also be computed using single
parameter optimization at this step. However, doing so requires that the least-squares
problem be solved for every change in value of $\sigma^2$.

\subsection{Random effect covariance priors}
\label{sec:bmath:covPriors}

The first concern that arises when placing a prior over the covariance
of the random effects is the structure of that matrix. While we have
consistently written simply `$\Sigma$' as if it were a monolithic
construct, in reality this matrix consists of block repetitions of
individual submatrices along its diagonal. If there are $K$ grouping
factors with $q_k$ varying coefficients and $J_k$ different groups at
level $k$, then we have $\Sigma_1, \ldots, \Sigma_k$ distinct covariance
matrices and $\Sigma = \mathrm{diag}_{k=1}^K \left( I_{J_k}\otimes
  \Sigma_k \right)$. In words, $\Sigma$ consists of $\Sigma_1$
repeated $J_1$ times along the diagonal, then $\Sigma_2$ repeated
$J_2$ times, and so forth. A prior over $\Sigma$ is in reality a prior
over $\Sigma_1, \ldots, \Sigma_k$, or equivalently a
reparameterization of these matrices. We denote the free parameters of
these matrices as $\theta$. For simplicity, we only consider covariance
priors that factor independently across these submatrices, although
joint priors remain an avenue for future work.

An examination of the profiled likelihood (equation
\ref{eq:profiledLikelihood}) or REML objective function (equation
\ref{eq:remlObjective}) demonstrates that priors on $\Sigma$
independent of the other parameters can be applied
directly to the covariance of the random effects - divided by the
residual variance - by simply adding a penalty term. For example, in the profiled posterior

\begin{align*}
  p(\Sigma(\theta) \mid y; \hat\beta, \hat\sigma^2) & \propto 
  p(y; \hat\beta, \hat\sigma^2, \Sigma(\theta)) {\color{blue} p(\Sigma(\theta))},  \\
  & \propto 
  \left\{\hat{\sigma}^2(\theta)^{-N/2}|L_Z(\theta)|^{-1} \right\}p(\Sigma(\theta)),
\end{align*}

\noindent the first term on the right-hand side, the profiled
likelihood, is calculated exactly as before.

The situation is more complicated when the prior is
to be applied in an absolute sense, that is not scaled by the
residual variance. In particular, this situation arises when substantive prior
information exists. As will be shown, imposing a prior of
this sort that is independent of the other parameters is actually equivalent to placing a joint prior on
$\sigma^2$ and $\Sigma$. If profiling is to be preserved, 
covariance priors not on the common scale are restricted to cases similar
to those outlined for the residual variance. It is important to
reiterate that the discussion that follows is entirely limited to this
case, and that scale-free optimization is trivial in comparison.

As an example of the problems involved, consider a set of data with a single grouping
factor. For notational simplicity, we will drop the $k=1$ subscript
when possible so that $q=q_1$ is the number of varying coefficients at
this level and $\tilde\Sigma= \sigma^2 \Sigma_1$ is the associated
absolute-scale covariance matrix. Suppose that it is desired to place an inverse Wishart prior on
$\tilde\Sigma$ with $\mu$ degrees of freedom and scale matrix
$\Psi$. Applying this to the likelihood (equation \ref{eq:likelihood})
produces the posterior density

\begin{multline*}
  p(\tilde\Sigma \mid y; \beta, \sigma^2) \propto
  (\sigma^2)^{-N/2}|L_Z|^{-1} {\color{blue}|\tilde\Sigma|^{-(\mu + q + 1)/2}}
  \exp\left\{-\frac{1}{2{\color{blue}\cdot 1}} \mathrm{tr}\left(\Phi{\color{blue}\tilde\Sigma^{-1}}\right)\right\}\times \\
  \exp\left\{-\frac{1}{2\sigma^2} \left[(\beta - \tilde\beta)^\top
      L_XL_X^\top (\beta - \tilde\beta) + \left\|\begin{bmatrix} y \\
          0 \end{bmatrix} -
        \begin{bmatrix} Z\Lambda & X \\ I_Q & 0 \end{bmatrix}
        \begin{bmatrix} u \\ \beta \end{bmatrix}\right\|^2\right]\right\}.
\end{multline*}

However, this equation is to be optimized point-wise. For the values
of $\hat\beta$, $\hat\sigma^2$, and $\widehat{\tilde\Sigma}$ that
maximize it, $\hat\beta$, $\hat\sigma^2$, and $\hat\Sigma_1 =
\widehat{\tilde\Sigma} / \hat\sigma^2$ achieve
the same value. That is, the right-hand side is the same if we instead
work with the equation:

\begin{multline*}
  p(\tilde\Sigma \mid y; \beta, \sigma^2) \propto 
  (\sigma^2)^{-(N+{\color{blue}\mu+q+1})/2}
  |L_Z|^{-1}{\color{blue}|\Sigma_1|^{-(\mu + q + 1)/2}} \times \\
  \exp\left\{-\frac{1}{2\sigma^2}\left[(\beta - \tilde\beta)^\top
      L_XL_X^\top (\beta - \tilde\beta) + \left\|\begin{bmatrix} y \\
          0 \end{bmatrix} -
        \begin{bmatrix} Z\Lambda & X \\ I_Q & 0 \end{bmatrix}
        \begin{bmatrix} u \\ \beta \end{bmatrix}\right\|^2 +
      \mathrm{tr} \left(\Phi{\color{blue}\Sigma_1^{-1}}\right)\right]\right\}.
\end{multline*}

Consequently, we have that an inverse Wishart prior on the unscaled
matrix $\tilde\Sigma$ with degrees of freedom $\mu$ and scale $\Psi$
is equivalent to the joint prior

\begin{align*}
  \sigma^2 \mid \Sigma_1 & \sim \mathrm{Inv-Gamma}
  \left(\mathrm{shape} = (\mu + q - 1) / 2, \text{ scale} =
    \mathrm{tr}\left(\Psi\Sigma_1^{-1}\right) / 2\right), \\
  \Sigma_1 & \sim \mathrm{Inv-Wishart}(\mu, \Psi). \\
\end{align*}

Doing similar calculations for the cases of a Wishart prior on $\tilde\Sigma$ and an
inverse Wishart on the unique positive definite square root of $\tilde\Sigma$
yield analogous schemes to the optimization of $\sigma^2$ in
section \ref{sec:bmath:residPrior}, with an induced gamma distribution on
$\sigma^2$ and inverse gamma distribution on $\sigma$
respectively. 

Taking into consideration these induced priors and the added complexity
of $\Sigma$ being comprised of multiple covariance sub-matrices,
straightforward profiling of $\sigma^2$ is constrained to the cases where the
totality of all induced priors yield a quadratic or linear optimization
problem. Specifically, inverse Wishart distributions applied to the
unscaled covariance matrices can be added in any quantity
desired, as the net effect is to add terms to the penalized residual
sum of squares. For any remaining grouping factors for which an
absolute-scaled prior is desired, either Wishart distributions on covariance
matrixes or inverse Wishart distributions on covariance square roots
must be chosen and applied consistently. To profile $\sigma^2$, the
various coefficients of the exponential terms are collected for each
optimization iteration by cycling through the grouping factors, and
the resulting linear or quadratic problem solved.

Finally, we briefly discuss improper Wishart-family priors. The principle
complication in modeling unscaled random effect covariances is in
the exponential term involving $\sigma^2$ introduced into the
likelihood. The polynomial term simply increases or decreases the
degrees of freedom. Thus, improper distributions that
eliminate the exponential term are trivial to include. These are the
limits of gamma/Wishart distributions as the scale tends to infinity
or the limits of the inverse as the scale tends to 0. Further allowing
improper degrees of freedom as well shows that the choice of prior
parameterization is irrelevant, that is Wishart priors on covariances, square
roots, or their inverses all yield penalties terms that are powers of
the determinant.

\subsection{Summary}

\begin{table}
  \begin{center}
  \begin{tabular}{ccc}
    Fixed Effects & Residual Variance & Random Effect Covariance \\
    $\beta$ & $\sigma^2$ & $\Sigma_k$ \\ \hline
    $\beta \mid \sigma^2 \sim \mathcal{N}(0, \sigma^2\Sigma_\beta)$ & 
    $\sigma^2 \sim \mathrm{Inv-Gamma}$ & $p(\Sigma_k)$ arbitrary \\
    & $\sigma^2 \sim \mathrm{Gamma}$ \\
    & $\sigma \sim \mathrm{Inv-Gamma}$ & $\tilde\Sigma_k =
    \sigma^2\Sigma_k$, \\
    & $\sigma^2 = \sigma^2_0 \text{ w.p.} 1$ & $\tilde\Sigma_k \overset{\rm ind}{\sim} {\rm Wish/Inv-Wish}$ \\
    & & $\tilde\Sigma_k^{1/2} \overset{\rm ind}{\sim} {\rm Inv-Wish}$
  \end{tabular}
\end{center}
  \caption{Priors for linear mixed models that can be fit into the
    two-stage profiling scheme of \pkg{lme4}.}
  \label{tab:easyPriors}
\end{table}

Table \ref{tab:easyPriors} lists the priors that in our discussion we have shown can be
fully profiled down to functions that are just the covariance of the
random effects. Provided that caution is exercised so
that the optimization of the residual variance is straightforward, an option can be selected
from each column. In all cases, the algorithm to calculate the
profiled objective function is given by:

\begin{enumerate}
\item Determine the maximizer of the joint density of the
  observations and the spherical random effects in $(u, \beta)$. In
  doing so, the integral of the joint density with respect to the
  random effects is effectively computed as well.
\item Plug in the joint mode of $\beta$ to the likelihood, as it is also the maximum
  likelihood/maximum a posteriori estimate.
\item Calculate the mode of $\sigma^2$ as the solution to a linear or
  quadratic problem, depending on the choice of priors. Point priors
  involve no optimization, and more complicated cases may require root-finding.
\item Plug in the estimate of $\sigma^2$ and numerically optimize the
  resulting function.
\end{enumerate}

\section{Generalized linear mixed model specification}
\label{sec:glmms}

As there are no obvious ways by which a generalized linear mixed model
can be profiled, \pkg{lme4} performs optimization for these models
numerically across the entire set of parameters. In order to apply priors on
these models as well, we write out their
specification. Generalized linear models principally differ from their
regular counterparts through the use of a link function, here denoted
`$g$', that maps linear combinations of coefficients to the expected
value of the response. In addition, the response given the random
effects is no longer assumed
Gaussian but instead merely a member of the the exponential family of
distributions. While omitting the specifics of this conditional distribution,
generalized linear mixed models can be written as

\begin{align*}
  \E[y \mid b; \beta, \Sigma] & = g^{-1}(X\beta + Zb), \\
  b & \sim \mathcal{N}(0, \Sigma),
\end{align*}

\noindent where $g^{-1}$ is the inverse of the link function and it is
to be understood as applied component-wise to the vector of linear predictors.

Given this specification, the parameters of the model are the
fixed effects, $\beta$, and the covariance of the random effects,
$\Sigma$. Like linear mixed models, $\Sigma$ is highly structured and
comprised of sub-matrices. Unlike linear mixed models, there is no scale
parameter/residual variance. The family associated with the response
may have additional parameters, but as of this writing the ``quasi''
options that include overdispersion are not supported by \pkg{lme4}.

\pkg{lme4} performs optimization for generalized linear mixed models
by assuming values for $\beta$ and $\Sigma$, approximating the
integral over the random effects, and returning a value for the
likelihood to a numeric optimizer. To apply a prior to either parameter,
it is sufficient to include a penalty term at this last step.

\section[blme overview]{\pkg{blme} overview}

We have written the software package \pkg{blme} for the \proglang{R}
statistical programming language that performs profiled posterior
maximization in linear and generalized linear mixed models. \pkg{blme}
is based on \pkg{lme4} and uses as much machinery of that package
possible. In particular, the efficient \proglang{C++} code and matrix
decompositions that underly an \pkg{lme4} model fit are also used by \pkg{blme}.

\subsection[Calling blme functions]{Calling \pkg{blme} functions}

\pkg{blme} was designed to be familiar to users of \pkg{lme4}. A
\code{bmerMod} S4 object extends the \code{merMod} class, and
consequently inherits all of the same functionality. In \pkg{lme4}, linear mixed
models are fit using the \code{lmer} function, while generalized
models use \code{glmer}. Fitting a model in \pkg{blme} is achived by
instead calling \code{blmer} or \code{bglmer}, 
modified versions of the original functions with new arguments.

The prototypes for \code{blmer} and \code{bglmer} are:

\begin{CodeChunk}
  \begin{Code}
blmer(formula, data, REML = TRUE, control = lmerControl(), start = NULL,
      verbose = 0L, subset, weights, na.action, offset, contrasts = NULL,
      devFunOnly = FALSE, 
      cov.prior = wishart, fixef.prior = NULL, resid.prior = NULL, ...)
  \end{Code}
  \begin{Code}
bglmer(formula, data, family = gaussian, control = glmerControl(),
       start = NULL,  verbose = 0L, nAGQ = 1L, subset, weights, na.action, 
       offset, contrasts = NULL, mustart, etastart, devFunOnly = FALSE, 
       cov.prior = wishart, fixef.prior = NULL, ...)
  \end{Code}
\end{CodeChunk}

\noindent For both, all but the last line are identical to their
\pkg{lme4} equivalent. The new arguments are
\code{cov.prior}, \code{fixef.prior}, and \code{resid.prior}. All
three apply to linear mixed models, while only the first two apply to
generalized ones, for the reasons discussed in section \ref{sec:glmms}.

The format for each new argument is analogous to that of a delayed
function call that is evaluated in a special environment. This is done so 
that priors can refer to particulars of the model without having these
explicitly defined at the calling level, for example allowing a default prior for random effect covariance
matrices that scales with the dimensions of the grouping
factors. Variables available at prior specification are:

\begin{enumerate}
\item \code{q.k} or \code{level.dim} - dimension of $\Sigma_k$, or how many
  coefficients vary at the $k$th level; covariance priors only
  \item \code{j.k} or \code{n.grps} - number of groups at the
  $k$th level; covariance priors only
\item \code{n} or \code{n.obs} - number of observations
\item \code{p} or \code{n.fixef} - number of fixed effects
\end{enumerate}

Exceptions to the function-call syntax are that 0-argument function invocations
can be expressed without the empty \code{()} parentheses, character
strings can be passed instead to aid in reuse, and for
multiple grouping factors \proglang{R} formula notation is abused to
specify different priors for different levels. In the case of multiple
priors for a single argument, the input should be a list. Examples and
descriptions follow in the relevant sections below.

Finally, note that the default random effect covariance prior is set
to \code{wishart} for both functions. This provides a
weakly-informative prior that mitigates the downward bias of
covariance estimates and guarantees that the result will be strictly
positive definite \TODO{REF}. For the sake of exposition, in the fixed effect and
conditional prior examples discussed below the covariance prior will be explicitly
disabled so that model being discussed differs from the \pkg{lme4}
version by only the prior under consideration.

\subsection{Fixed effect priors}
\label{sec:blme:fixefPriors}

Currently supported fixed effect priors are flat/\code{NULL}, multivariate
normal, and multivariate Student's $t$ distributions. Flat and normal
priors can be profiled without adding the fixed effects to the numeric
optimization parameter set. $t$ distributions require additional computation, 
and thus can be considerably slower to fit. In addition, $t$ priors
cannot be used when the \code{REML} is true. See
section \ref{sec:bmath:fixefPrior} for details.

To specify a fixed effect prior, pass to \code{blmer} or \code{bglmer}
a value for the \code{fixef.prior} argument of the format:

\begin{Code}
  distribution.name(options.list)
\end{Code}

\noindent where \code{distribution.name} can be \code{normal} or
\code{t}. An argument of \code{NULL}/\code{flat} is equivalent to a
flat prior or no
penalty term. Options and defaults are:

\begin{enumerate}
\item
\verb?normal(sd = c(10, 2.5), cov, common.scale = TRUE)?  \begin{enumerate}
  \item \code{sd} - a vector of standard deviations. If length 1,
    the value is reused for all components. If length 2, the first
    item applies to the first fixed effect (typically the intercept
    term), while the second is reused. Otherwise must be of length
    equal to the number of fixed effects.
  \item \code{cov} - either a vector of variances of length equal to
    the number of fixed effects or a positive definite matrix of
    appropriate size. Only one of \code{sd} or \code{cov} should be
    specified.
  \item \code{common.scale} - a logical that when true implies that prior
    is of the form $\beta\mid \sigma^2 \sim \mathcal{N}(0,
    \sigma^2\Sigma_\beta)$. When false, $\beta \sim \mathcal{N}(0,
    \Sigma_\beta)$. Only applies to linear mixed models.
  \end{enumerate}
\item
\verb?t(df = 3, scale = c(10^2, 2.5^2), common.scale = TRUE)?  \begin{enumerate}
  \item \code{df} - a positive scalar signifying the degrees of freedom.
  \item \code{scale} - determines the scale matrix. If is a scalar,
    the scale is that value times the identity matrix. For a length of
    2 the second value is placed along the diagonal for all intercepts while the
    off-diagonals are set to 0. For a length equal to the number of fixed effects,
    the values are placed directly on the diagonal. Otherwise, a
    positive definite matrix of the appropriate size can be given.
  \item \code{common.scale} a logical that when true implies that prior
    is of the form $\beta\mid \sigma^2 \sim t_\nu(0,
    \sigma^2\Sigma_\beta)$. When false, $\beta \sim t_\nu(0,
    \Sigma_\beta)$. Only applies to linear mixed models.
  \end{enumerate}
  The $t$ prior used has the density
  \begin{equation*}
    \beta \sim t_\nu(0, \Sigma_\beta) \Rightarrow p(\beta) =
    \dfrac{\Gamma\left((\nu + P)/2\right)}
    {\Gamma(\nu/2) (\nu\pi)^{P/2}|\Sigma_\beta|^{1/2} \left[
        1 + \frac{1}{\nu}\beta^\top
        \Sigma_\beta^{-1}\beta\right]^{(\nu + P)/2}},
  \end{equation*}
  \noindent where $P$ is the number of fixed effects. \TODO{reference}.
\end{enumerate}

\subsection{Fixed effect examples}

For the purposes of illustration, suppose that we have defined in the
global environment a response variable \code{y}, a predictor
\code{x}, and a grouping factor \code{g}.

\subsubsection*{Default normal prior}

The following applies a normal prior using the default
hyperparameters, while disabling the default random effect covariance prior.

\begin{Code}
blmer(y ~ 1 + x + (1 + x | g), cov.prior = NULL,
      fixef.prior = normal)
\end{Code}

\subsubsection*{Normal prior with specific scales}

To do ridge regression, one can cross-validate to find the optimal
\code{lambda}:

\begin{Code}
blmer(y ~ 1 + x + (1 + x | g), cov.prior = NULL,
      fixef.prior = normal(sd = 1 / sqrt(lambda)))
\end{Code}

\subsubsection*{Normal prior with substantive information}

With prior information for the scale of the fixed effects, it becomes
necessary to model $\beta$ without using the common scale. Assuming
that this information is in the matrix \code{Sigma.beta}, the
\code{blmer} call to incorporate it is:

\begin{Code}
blmer(y ~ 1 + x + (1 + x | g), cov.prior = NULL,
      fixef.prior = normal(cov = Sigma.beta, common.scale = FALSE))
\end{Code}

\subsubsection*{$t$ prior for
  logistic regression}

\citet{gelman:2008:weakly_informative} recommend a Cauchy prior for logistic
regression. \code{bglmer} fits an analogous mixed model by using an appropriately
parameterized $t$ distribution:

\begin{Code}
bglmer(y ~ 1 + x + (1 + x | g), cov.prior = NULL,
       family = binomial,
       fixef.prior = t(df = 1, scale = 2.5^2))
\end{Code}

\subsection{Residual variance priors}
\label{sec:blme:residPrior}

A prior on the residual variance/common scale is specified in the same
format as for fixed effects, but instead using the \code{resid.prior} argument to
\code{blmer}. As discussed in section \ref{sec:glmms}, the generalized
linear mixed models fit by \pkg{lme4} do not have a scale parameter.

The named distributions that can be used are: \code{gamma},
\code{invgamma}, \code{point}, and \code{NULL}/\code{flat}. Options and defaults are:

\begin{enumerate}
\item
\verb?gamma(shape = 0, rate = 0, posterior.scale = "var")?  \begin{enumerate}
  \item \code{shape} - non-negative scalar. For a shape of 0, the
    prior is improper.
  \item \code{rate} - non-negative scalar. For a rate of 0, the prior
    is improper.
  \item \code{posterior.scale} - one of \code{"sd"} or
    \code{"var"}, corresponding to $\sigma \sim \Gamma({\rm shape},
    {\rm rate})$ and $\sigma^2 \sim \Gamma({\rm shape}, {\rm rate})$ respectively.
  \end{enumerate}
  The default setting applies the improper prior $p(\sigma^2)\propto
  (\sigma^2)^{-1}$.
\item
\verb?invgamma(shape = 0, scale = 0, posterior.scale = "var")?  \begin{enumerate}
  \item \code{shape} - non-negative scalar. For a shape of 0, the
    prior is improper.
  \item \code{scale} - non-negative scalar. For a scale of 0, the prior
    is improper.
  \item \code{posterior.scale} - one of \code{"sd"} or
    \code{"var"}, having the same interpretation as for \code{gamma}.
  \end{enumerate}
\item
\verb?point(value = 1, posterior.scale = "sd")?\\
Fixes $\sigma^2$ to a specific value, and is equivalent to a
  point-mass prior.
  \begin{enumerate}
  \item \code{value} - a positive scalar.
  \item \code{posterior.scale} - one of \code{"sd"} or
    \code{"var"}, having the same interpretation as for \code{gamma}.
  \end{enumerate}
\end{enumerate}

\subsection{Residual variance example}

A common use for residual variance priors is when the observations
have known but unequal residual variances. Assuming that these are
defined in the variable \code{resid.var}, this model can be fit in
\code{blmer} by the call:

\begin{Code}
blmer(y ~ 1 + x + (1 + x | g), cov.prior = NULL,
      weights = 1 / resid.var,
      resid.prior = point)
\end{Code}

\subsection{Random effect covariance priors}
\label{sec:blme:covPriors}

As with a mixed effect model there can be multiple
grouping factors each with its own distinct covariance matrix,
\pkg{blme} supports the specification of a default prior as well as
priors specific to named levels. To apply a prior to single level, the
input should be of the form:

\begin{Code}
  grouping.name ~ distribution.name(options.list)
\end{Code}

Conversely, to specify a default prior that applies to all grouping
factors, the format is simply:
\code{distribution.name(options.list)}. The distributions and their
options are:

\begin{enumerate}
\item
\verb?gamma(shape = 2.5, rate = 0, common.scale = TRUE, posterior.scale = "sd")?\\  Can only be used with univariate grouping factors.
  \begin{enumerate}
  \item \code{shape} - a non-negative scalar. For a shape of 0, the
    prior is improper.
  \item \code{rate} - a non-negative scalar. For a rate of 0, the
    prior is improper.
  \item \code{common.scale} - a logical that when true indicates that
    the prior should be applied to the random effect covariance divided by
    the residual variance while when false the prior is applied to the covariance
    in the absolute sense. Note that this is slightly different than
    the interpretation for fixed effects, as there it influenced the
    scale and not the parameters themselves. Only applies to linear
    mixed models. See sections
    \ref{sec:math:commonScale} and \ref{sec:bmath:covPriors}.
  \item \code{posterior.scale} - one of \code{"sd"} or \code{"var"}
    corresponding to $\sqrt\Sigma_k \sim \Gamma({\rm shape}, {\rm rate}$
    and $\Sigma_k \sim \Gamma({\rm shape}, {\rm rate}$ respectively.
  \end{enumerate}
\item
\verb?invgamma(shape = 0.001, scale = shape + 0.05, common.scale = TRUE,?\\
\verb?         posterior.scale = "var")?  \begin{enumerate}
  \item \code{shape} - a non-negative scalar. For a shape of 0, the
    prior is improper.
  \item \code{scale} - a non-negative scalar. For a scale of 0, the
    prior is improper.
  \item \code{common.scale} - logical with similar interpretation as
    in the \code{gamma} case.
  \item \code{posterior.scale} - one of \code{"sd"} or \code{"var"}
    having a similar interpretation as for \code{gamma}.
  \end{enumerate}
\item
\verb?wishart(df = level.dim + 2.5, scale = Inf, common.scale = TRUE,?\\
\verb?        posterior.scale = "cov")?\begin{enumerate}
  \item \code{df} - a scalar greater than or equal to the number of
    varying coefficients for a level, minus 1. When equal to the lower
    bound, the prior is improper.
  \item \code{scale} - a single value, a vector of length equal to the
    grouping factor dimension, or an appropriately sized positive
    definite matrix. For the first case, the value is multiplied by
    the identity matrix; for the second, the vector is made into a
    diagonal matrix. \code{Inf} is allowed, and yields an improper distribution.
  \item \code{common.scale} - logical with similar interpretation as
    in the \code{gamma} case.
  \item \code{posterior.scale} - one of \code{"sqrt"} or \code{"var"}
    having a similar interpretation as for \code{gamma}. Note that the
    unique matrix square root can be expensive to compute, as the
    optimization is not performed in this parameterization.
  \end{enumerate}
\item
\verb?invwishart(df = level.dim - 0.998, scale = diag(df + 0.1, level.dim),?\\
\verb?           common.scale = TRUE, posterior.scale = "cov")?  \begin{enumerate}
  \item \code{df} - a scalar greater than or equal to the number of
    varying coefficients for a level, minus 1. When equal to the lower
    bound, the prior is improper.
  \item \code{scale} - a single value, a vector of length equal to the
    grouping factor dimension, or an appropriately sized positive
    definite matrix. For the first case, the value is multiplied by
    the identity matrix; for the second, the vector is made into a
    diagonal matrix. \code{0} is allowed, and yields an improper distribution.
  \item \code{common.scale} - logical with similar interpretation as
    in the \code{gamma} case.
  \item \code{posterior.scale} - one of \code{"sqrt"} or \code{"var"}
    having a similar interpretation as for \code{wishart}.
  \end{enumerate}
\item
\verb?custom(fn, chol = FALSE, common.scale = TRUE, scale = "none")?\\ Can be used to apply an arbitrary function as a penalty
  \begin{enumerate}
  \item \code{fn} - function to be used as a prior. Must take an
    argument as specified by \code{chol} and return a value as
    specified by \code{scale}.
  \item \code{chol} - a logical indicating whether or not \code{fn}
    expects a covariance matrix (\code{FALSE}) or a {\em right}
    Cholesky factor (\code{TRUE}).
  \item \code{common.scale} - logical with same interpretation as
    other cases. When \code{FALSE}, it is unknown if the residual
    variance can be profiled and is thus is added to the set of
    parameters for numeric optimization.
  \item \code{scale} - one of \code{"none"}, \code{"log"}, or
    \code{"dev"} corresponding to \code{fn} returning values on the scales
   $p(\Sigma)$, $\log p(\Sigma)$, and $-2 \log p(\Sigma)$ respectively.
 \end{enumerate}
\end{enumerate}

\subsection{Covariance examples}

For the following, we now assume that there are two grouping factors,
\code{g.1} and \code{g.2}.

\subsubsection*{\pkg{lme4} fit}

By default, \pkg{blme} applies a Wishart prior to the covariance of
the random effects. By suppressing this, the following two calls are equivalent:

\begin{Code}
lmer (y ~ 1 + x + (1 | g.1) + (1 + x | g.2))
blmer(y ~ 1 + x + (1 | g.1) + (1 + x | g.2),
      cov.prior = NULL)
\end{Code}

\subsubsection*{Univariate, default prior, standard deviation scale}

The follow places a prior on the standard deviation of the
contributions to the intercept for the first grouping factor:

\begin{Code}
blmer(y ~ 1 + x + (1 | g.1) + (1 + x | g.2),
      cov.prior = gamma)
\end{Code}

As the prior was univariate, it does not extend to the second grouping
factor. Instead, this remains unmodeled. If a third level with a
single varying coefficient existed, the gamma prior would apply to
that as well.

\subsubsection*{Multivariate, default prior, variance scale}

Using a Wishart as a default specializes down to a gamma for a
univariate case and consequently applies to both grouping factors. The
default for the Wishart is to use a posterior scale of a covariance,
unlike the previous example.

\begin{Code}
blmer(y ~ 1 + x + (1 | g.1) + (1 + x | g.2),
      cov.prior = wishart)
\end{Code}

\subsubsection*{Named grouping factors}

We can mix the above examples by naming the grouping factors.

\begin{Code}
blmer(y ~ 1 + x + (1 | g.1) + (1 + x | g.2),
      cov.prior = list(g.1 ~ gamma, g.2 ~ wishart))
\end{Code}

\subsubsection*{Default priors with options}

As with univariate families there are no complications with level
dimensions, it is easy to specify a default that applies in more than
one case. For the following, we no longer model the random effects of
the second grouping factor as having a varying slope.

\begin{Code}
blmer(y ~ 1 + x + (1 | g.1) + (1 | g.2),
      cov.prior = gamma(shape = 3, rate = 1, posterior.scale = 'var'))
\end{Code}

\subsubsection*{Fixed hyperparameters}

The parent of the evaluating environment for the prior creation
functions is set to the one which calls \code{blmer} so it is possible to
use variables defined there.

\begin{Code}
g.2.cov <- matrix(c(1, 0.2, 0.2, 1), 2)
blmer(y ~ 1 + x + (1 | g.1) + (1 + x | g.2),
      cov.prior = g.2 ~ wishart(scale = g.2.cov))
\end{Code}

\subsubsection*{Complex expressions}

When used with default settings, the Wishart has its degrees of freedom set to
the number of coefficients varying at a grouping factor plus 2.5. The
following applies a default prior that penalizes the covariance for each level
by a term that is $|\Sigma_k|^{1/2}$. This results in a linear term on
the standard deviation when there is only one varying coefficient.

\begin{Code}
blmer(y ~ 1 + x + (1 | g.1) + (1 + x | g.2),
      cov.prior = wishart(df = level.dim + 2))
\end{Code}

For a Wishart distribution with $\nu$ degrees of freedom, $q_k$
dimension, and a scale matrix of $\Psi$, its mode is given by $(\nu -
q_k - 1)\Psi$. We can thus construct a proper verison of the Wishart
prior that is parameterized by its degrees of freedom and mode and use
this as a default.

\begin{Code}
prior.mode <- 1e-4
blmer(y ~ 1 + x + (1 | g.1) + (1 + x | g.2),
      cov.prior = wishart(scale = diag(prior.mode, q.k) / (df - q.k - 1)))
\end{Code}

\subsubsection*{Custom prior}

\TODO{REF} recommends a half-Cauchy prior for variance components in
linear mixed models. For univariate grouping factors,

\begin{Code}
penaltyFn <- function(sigma) dcauchy(sigma, 0, 10, log = TRUE)
blmer(y ~ 1 + x + (1 | g.1) + (1 | g.2),
      cov.prior = custom(penaltyFn, chol = TRUE, scale = "log"))
\end{Code}

\subsection[blme output]{\code{blme} output}

\pkg{blme} functions return objects that inherit from the \pkg{lme4}
base classes. For linear models, the result is a \code{blmerMod}
derived from \code{lmerMod}, while in the general case the result is
of class \code{bglmerMod}, derived from \code{glmerMod}.

The \code{summary} function for each highlights the main
differences from \pkg{lme4}.

\begin{CodeChunk}
\begin{CodeInput}
> summary(fm1 <- lmer(Reaction ~ Days + (Days | Subject), sleepstudy))
\end{CodeInput}
\begin{CodeOutput}
Linear mixed model fit by REML ['lmerMod']
Formula: Reaction ~ Days + (Days | Subject)
   Data: sleepstudy

REML criterion at convergence: 1743.6

Scaled residuals: 
    Min      1Q  Median      3Q     Max 
-3.9536 -0.4634  0.0231  0.4634  5.1793 

Random effects:
 Groups   Name        Variance Std.Dev. Corr
 Subject  (Intercept) 612.09   24.740       
          Days         35.07    5.922   0.07
 Residual             654.94   25.592       
Number of obs: 180, groups:  Subject, 18

Fixed effects:
            Estimate Std. Error t value
(Intercept)  251.405      6.825   36.84
Days          10.467      1.546    6.77

Correlation of Fixed Effects:
     (Intr)
Days -0.138
\end{CodeOutput}
\begin{CodeInput}
> summary(bm1 <- blmer(Reaction ~ Days + (Days | Subject), sleepstudy))
\end{CodeInput}
\begin{CodeOutput}
Cov prior  : Subject ~ wishart(df = 4.5, scale = Inf, posterior.scale = cov,
           : common.scale = TRUE)
Prior dev  : 2.4021

Linear mixed model fit by REML ['blmerMod']
Formula: Reaction ~ Days + (Days | Subject)
   Data: sleepstudy

REML criterion at convergence: 1749.5

Scaled residuals: 
    Min      1Q  Median      3Q     Max 
-4.2393 -0.4485 -0.0003  0.4858  5.3884 

Random effects:
 Groups   Name        Variance Std.Dev. Corr 
 Subject  (Intercept) 840.64   28.994        
          Days         97.29    9.864   -0.30
 Residual             606.69   24.631        
Number of obs: 180, groups:  Subject, 18

Fixed effects:
            Estimate Std. Error t value
(Intercept)  251.405      7.638   32.91
Days          10.467      2.411    4.34

Correlation of Fixed Effects:
     (Intr)
Days -0.362
\end{CodeOutput}
\end{CodeChunk}

Aside from the different point estimates, the first lines of output a)
provide a textual summary of the priors used and b) quantify the
contribution of the priors to the objective function. This information
can also be accessed directly from the fitted result.

\begin{Schunk}
\begin{Sinput}
> bm1@priors
\end{Sinput}
\begin{Soutput}
$covPriors
$covPriors[[1]]
An object of class "bmerWishartDist"
Slot "df":
[1] 4.5

Slot "R.scale.inv":
     [,1] [,2]
[1,]    0    0
[2,]    0    0

Slot "log.det.scale":
[1] Inf

Slot "posteriorScale":
[1] "cov"

Slot "commonScale":
[1] TRUE
\end{Soutput}
\begin{Sinput}
> round(bm1@devcomp$cmp, 3)
\end{Sinput}
\begin{Soutput}
      ldL2      ldRX2       wrss       ussq      pwrss      drsum       REML 
    96.586      7.130  95968.324  12022.569 107990.893         NA   1749.486 
       dev    sigmaML  sigmaREML    penalty 
        NA     24.494     24.631      2.402 
\end{Soutput}
\end{Schunk}

This enables a direct comparison of the distance of the fit from the
maximum likelihood or REML estimate.


\section{Examples}

\subsection{Complete separation for mixed models with binary outcomes}
\label{sec:eg:fixef}


\citet{mckeon:2012:defender} report an experiment on the frequency of
predation of coral reefs and the effect of cohabiting crustaceans who
defending the colony. The data consist of four treatment groups corresponding
to combinations of symbiotic species, and for each treatment pairs of observations are
made across 10 temporal blocks. Exploring the data, we find:

\begin{Schunk}
\begin{Sinput}
> culcita[1:5,]
\end{Sinput}
\begin{Soutput}
  block predation  ttt
1     1         0 none
2     1         1 none
3     2         1 none
4     2         1 none
5     3         1 none
\end{Soutput}
\begin{Sinput}
> table(culcita$block)
\end{Sinput}
\begin{Soutput}
 1  2  3  4  5  6  7  8  9 10 
 8  8  8  8  8  8  8  8  8  8 
\end{Soutput}
\begin{Sinput}
> levels(culcita$ttt)
\end{Sinput}
\begin{Soutput}
[1] "none"   "crabs"  "shrimp" "both"  
\end{Soutput}
\end{Schunk}

Using \code{glmer} we can fit a generalized linear mixed model with binomial response and
a logistic link function that allows for a random effect corresponding
to the temporal block. Following
\citet{gelman:2008:weakly_informative}, we standardize binary inputs by
centering them to have a mean of 0 and store the result in
\code{culcita.z}. For compactness, we use the \code{display} function
from the \pkg{arm} package to show the fit.

\begin{Schunk}
\begin{Sinput}
> m1 <- glmer(predation ~ tttcrabs + tttshrimp + tttboth + (1 | block),
+             culcita.z, family = binomial)
> display(m1)
\end{Sinput}
\end{Schunk}
\begin{Schunk}
\begin{Soutput}
glmer(formula = predation ~ tttcrabs + tttshrimp + tttboth + 
    (1 | block), data = culcita.z, family = binomial, nAGQ = 10)
            coef.est coef.se
(Intercept)  1.60     1.21  
tttcrabs    -3.75     1.38  
tttshrimp   -4.36     1.45  
tttboth     -5.55     1.52  

Error terms:
 Groups   Name        Std.Dev.
 block    (Intercept) 3.50    
 Residual             1.00    
---
number of obs: 80, groups: block, 10
AIC = 70.3, DIC = 60.3
deviance = 60.3 
\end{Soutput}
\end{Schunk}

A closer look at the data reveals that the model's stability hinges on
the presence of a single observation. In fact, in the control group there are
only two observations that fail to exhibit predation, one in the first
block and one in the 10th.

\begin{Schunk}
\begin{Sinput}
> with(culcita, sum(predation == 0 & ttt == "none"))
\end{Sinput}
\begin{Soutput}
[1] 2
\end{Soutput}
\end{Schunk}

If for the control no observations were positive, then the likelihood
could be made arbitrarily large by increasing the value of the
control group's coefficient, {\em i.e.} quasi-complete separation. On
the other hand, in the first temporal block only one observation
shows predation. Furthermore, it contains one of the two observations
that are preventing separation.

\begin{Schunk}
\begin{Sinput}
> subset(culcita, block == 1)
\end{Sinput}
\begin{Soutput}
   block predation    ttt
1      1         0   none
2      1         1   none
21     1         0 shrimp
22     1         0 shrimp
41     1         0  crabs
42     1         0  crabs
61     1         0   both
62     1         0   both
\end{Soutput}
\end{Schunk}

These two circumstances combine so that removing the negative
case for the control group in the 10th temporal block nearly makes the
problem ill-posed. For consistency, we also remove its pair and re-standardize the
result in the variable \code{culcitaSep.z}.  The
fitted generalized linear mixed model yields substantially different
estimates.

\begin{Schunk}
\begin{Sinput}
> m2 <- glmer(predation ~ tttcrabs + tttshrimp + tttboth + (1 | block),
+             culcitaSep.z, family = binomial)
> display(m2)
\end{Sinput}
\end{Schunk}

\begin{Schunk}
\begin{Soutput}
glmer(formula = predation ~ tttcrabs + tttshrimp + tttboth + 
    (1 | block), data = culcitaSep.z, family = binomial, nAGQ = 10)
            coef.est coef.se
(Intercept)   5.19     4.22 
tttcrabs    -15.47    12.63 
tttshrimp   -17.46    12.60 
tttboth     -20.50    12.56 

Error terms:
 Groups   Name        Std.Dev.
 block    (Intercept) 11.75   
 Residual              1.00   
---
number of obs: 78, groups: block, 10
AIC = 49.1, DIC = 39.1
deviance = 39.1 
\end{Soutput}
\end{Schunk}

For example, the fixed effect for the intercept jumped from
1.6 to 5.19
corresponding to a change in baseline probability of 
0.83 to 
0.99, most of the reported statistical
signifiance is lost, and the standard deviation of the random effects
went from
3.5 to
11.75.

While this is not strictly quasi-complete separation, the
situation is similar. The joint distribution of the observations and the
random effects can be made large by making the intercept a large
positive value and the random effect for the first block highly
negative. The extent to which this is possible depends on the
covariance of the random effects, which in turn is inflated. The model
only remains identifiable because the covariance cannot increase indefinitely.

While not as good as having the full data, with \code{bglmer} it is
possible to at least obtain sensible estimates. \citet{gelman:2008:weakly_informative} recommends the use
of a Cauchy prior with a scale between 0.75 and 2.5 for non-hierarchical logistic
regressions. With the knowledge that the problem is biased towards
large parameter values, we opt for the smaller scale parameter.

\begin{Schunk}
\begin{Sinput}
> m3 <- bglmer(predation ~ tttcrabs + tttshrimp + tttboth + (1 | block),
+              culcitaSep.z, family = binomial,
+              cov.prior = NULL, fixef.prior = t(1, 0.75))
> display(m3)
\end{Sinput}
\end{Schunk}

\begin{Schunk}
\begin{Soutput}
bglmer(formula = predation ~ tttcrabs + tttshrimp + tttboth + 
    (1 | block), data = culcitaSep.z, family = binomial, nAGQ = 10, 
    cov.prior = NULL, fixef.prior = t(1, 0.75))
            coef.est coef.se
(Intercept)  1.88     1.51  
tttcrabs    -4.07     1.53  
tttshrimp   -4.81     1.64  
tttboth     -6.32     1.78  

Error terms:
 Groups   Name        Std.Dev.
 block    (Intercept) 4.40    
 Residual             1.00    
---
number of obs: 78, groups: block, 10
AIC = 53.9, DIC = 43.9
deviance = 43.9 
\end{Soutput}
\end{Schunk}

The point estimates are all still greater in magnitude than the
maximum likelihood ones for the full data, but are now on the same
scale. This can be visualized in figure
\ref{fig:culcita}, which shows the profiled objectives for the three models as functions of the
standard deviation of the random effects alone.


\begin{figure}[t]
  \centering \includegraphics{culcita}
  \caption{Objective functions for the three models of
    \ref{sec:eg:fixef}. The $x$ axis corresponds to the standard deviation of
    the random effects, denoted $\sigma_u$. Fixed effects have been
    profiled out numerically. For the likelihood with complete data
    (\code{m1}) and the posterior with a Cauchy prior on the fixed
    effects but incomplete
    data (\code{m3}), the curves rescaled so that they roughly integrate to
  1. For \code{m2}, the rescaling was chosen to facilitate visual comparison.}
  \label{fig:culcita}
\end{figure}

\section{References}
\bibliography{blme}

\section{Appendix}
\label{sec:appendix}

\subsection[The joint mode in u and beta]{The joint mode in $u$ and $\beta$}
\label{sec:appendix:jointMode}

Here we write out expressions to find the mode of the joint density of
$y$ and $u$ (equation \ref{eq:jointDensity}) in $u$ and $\beta$. If we
write $z = \begin{bmatrix}y \\ 0 \end{bmatrix}$,
$V = \begin{bmatrix} Z\Lambda & X \\ I_Q & 0 \end{bmatrix}$,
and $\gamma = \begin{bmatrix} u \\ \beta \end{bmatrix}$ then we
would have a normal equations problem with the maximizer $\hat{\gamma} =
(V^\top V)^{-1} V^\top z$. The goal is
to do this calculation while preserving the sparsity of $Z$ and
$\Lambda$, which requires operating block-wise.

Utilizing the decomposition from equation \ref{eq:blockDecomp} and
recalling our notation that $\tilde{u}$ and $\tilde{\beta}$ denote the
mode, we have

\begin{alignat*}{3}
  \begin{bmatrix} \tilde{u} \\ \tilde{\beta} \end{bmatrix} & =
  \left(\begin{bmatrix} L_Z & 0 \\ L_{ZX} & L_X \end{bmatrix}
    \begin{bmatrix} L_Z^\top & L_{ZX}^\top \\ 0 & L_X^\top \end{bmatrix}\right)^{-1}
  \begin{bmatrix}\Lambda^\top Z^\top & I_Q \\ X^\top &
    0 \end{bmatrix} \begin{bmatrix}y \\ 0 \end{bmatrix}, \\
%% line break
  & = \begin{bmatrix} L_Z^{-\top} & -L_Z^{-\top}L_{ZX}^\top
    L_X^{-\top} \\ 0 & L_X^{-\top} \end{bmatrix}
  \begin{bmatrix} L_Z^{-1} & 0 \\
    -L_X^{-1}L_{ZX}L_Z^{-1} & L_X^{-\top} \end{bmatrix}
  \begin{bmatrix} \Lambda^\top Z^\top y \\ X^\top y \end{bmatrix}, \\
%% line break
  & = \begin{bmatrix} L_Z^{-\top} & -L_Z^{-\top}L_{ZX}^\top
    L_X^{-\top} \\ 0 & L_X^{-\top} \end{bmatrix}
  \begin{bmatrix}
    L_Z^{-1} \Lambda^\top Z^\top y \\
    L_X^{-1}\left(X^\top y - L_{ZX}L_Z^{-1}\Lambda^\top Z^\top y\right)
  \end{bmatrix} \\
%% line break
  & = \begin{bmatrix} L_Z^{-\top} & -L_Z^{-\top}L_{ZX}^\top
    L_X^{-\top} \\ 0 & L_X^{-\top} \end{bmatrix}
  \begin{bmatrix}
    u' \\ L_X^{-1} \left(X^\top y - L_{ZX}u'\right) \end{bmatrix}
  & \text{for } u' = L_Z^{-1}\Lambda^\top Z^\top y, \\
%% line break
  & = \begin{bmatrix} L_Z^{-\top} & -L_Z^{-\top}L_{ZX}^\top
    L_X^{-\top} \\ 0 & L_X^{-\top} \end{bmatrix}
  \begin{bmatrix} u' \\ \beta' \end{bmatrix} & \text{for } \beta' =
  L_X^{-1}\left(X^\top y - L_{ZX} u'\right), \\
%% line break
  & = \begin{bmatrix} L_Z^{-\top} \left( u' - L_{ZX}^\top L_X^{-\top}
      \beta' \right) \\ L_X^{-\top}\beta' \end{bmatrix}.
\end{alignat*}

\noindent To find the joint mode, one then finds in order $u', \beta',
\tilde{\beta},$ and $\tilde{u}$. Every operation involving $L_Z$ and
$Z\Lambda$ can be done efficiently thanks to sparsity.

\begin{Schunk}
\begin{Sinput}
> ls()
\end{Sinput}
\begin{Soutput}
 [1] "as.matrix.VarCorr" "commaColumnWrap"   "common_dir"       
 [4] "dataDir"           "defaultImgHeight"  "defaultImgWidth"  
 [7] "defaultPars"       "display"           "fround"           
[10] "functionToString"  "hist.default"      "imgDir"           
[13] "makeOrCheckDir"    "oldPars"           "pfround"          
[16] "reinstallLibrary"  "reloadLibrary"     "reup"             
[19] "sourceDir"         "unloadLibrary"    
\end{Soutput}
\end{Schunk}
\end{document}
